\input{../../common/preamble.tex}

\begin{document}
\title{Supercomputers -- Problem set 2}
\author{Einar Baumann}
\maketitle

\section{Exercise 1: Processor data caches} % (fold)
\label{sec:exercise_1}
\begin{quotation}{\itshape
  The previous supercomputer at NTNU, \texttt{njord}, was based on the POWER5 dual-core chip. The two processors on a single chip each had a private L1 data cache of size 32 kB (kilobytes), a shared L2 cache of size 1.875 MB (megabytes), and an off-chip L3 cache of size 36 MB. Assume that we need to store floating point numbers in double precision. 

  \begin{itemize}
    \item How many floating point numbers can fit in each of the caches?
    \item What is the dimension of the largest square matrix that we can fit in each cache? Compare this with Exercise 6  in Problem Set 1.
  \end{itemize}
}\end{quotation}

The number of double precision (64-bit, or 8-byte) floats that can fit in a cache is given by the size of the cache in bytes divided by 8.

The size of an $n$-dimensional matrix is given by $n^2$, so the largest $n$ that can fit in each cache is given by the square root of the maximum number of floats the cache can fit.

\begin{center}
  \begin{tabular}{llll}
  \toprule 
  \textsc{Chache} & \textsc{Size} & \# \textsc{64-bit floats} & $n_{max}$  \\
  \midrule
  L1 & 32 kB    & $4,000$     & $63$ \\
  L2 & 1.875 MB & $234,375$   & $484$ \\
  L3 & 36 MB    & $4,500,000$ & $2121$ \\
  \bottomrule
  \end{tabular}
\end{center}
% section exercise_1 (end)


\section{Exercise 2: Communication within circuits} % (fold)
\label{sec:exercise_2}

% section exercise_2 (end)

\end{document}