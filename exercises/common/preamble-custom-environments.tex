%******************************************************************************
% Custom environment definitions. 
% REQUIRES the `color' package.
%******************************************************************************

\usepackage[amsmath,thmmarks,framed]{ntheorem}
\usepackage{framed}
\definecolor{env-gray}{rgb}{0.6,0.6,0.6}

% Examples environment
\theoremstyle{break}
\theoremframepreskip{1.5em}
\theoremframepostskip{1.5em}
\newframedtheorem{example}{\protect\examplename}[section]

% Remarks environment
\theoremstyle{plain}
\theorempreskip{1.5em}
\theorempostskip{1.5em}
\newtheorem{rem}{\protect\remarkname}[section]

% Definitions environment
\theoremstyle{plain}
\theoremframepreskip{1.5em}
\theoremframepostskip{1.5em}
\newframedtheorem{defn}{\protect\definitionname}


%---------------------------------
% Translations
%---------------------------------
\addto\captionsenglish{\renewcommand{\corollaryname}{Corollary}}
\addto\captionsenglish{\renewcommand{\definitionname}{Definition}}
\addto\captionsenglish{\renewcommand{\examplename}{Example}}
\addto\captionsenglish{\renewcommand{\remarkname}{Remark}}
\addto\captionsnorsk{\renewcommand{\corollaryname}{Korollar}}
\addto\captionsnorsk{\renewcommand{\definitionname}{Definisjon}}
\addto\captionsnorsk{\renewcommand{\examplename}{Eksempel}}
\addto\captionsnorsk{\renewcommand{\remarkname}{Merknad}}
\providecommand{\corollaryname}{Korollar}
\providecommand{\definitionname}{Definisjon}
\providecommand{\examplename}{Eksempel}
\providecommand{\remarkname}{Merknad}
