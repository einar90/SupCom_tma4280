% -*- root: ../supcom.tex -*-

\section{Matrix-matrix multiplication} % (fold)
\label{sec:matrix_matrix_multiplication}

Multiplying an $m\times n$ matrix $A$ with $m$ rows and $n$ columns and an $n\times l$ matrix $B$ with $n$ rows and $l$ columns produces an matrix $C$ with $m$ rows and $l$ columns. Each element of the matrix $C$ is calculated according to the formula:
\begin{equation}
  c_{ij} = \sum_{k=0}^{n-1} a_{ik} b_{kj}, \quad 0 \leq i < m, \; 0\leq j < l
\end{equation}
In other words $C$ is the result of the inner product of the corresponding row of the matrix $A$ and column of the matrix $B$.

This algorithm executes $m\cdot n\cdot l$ multiplications and the same number of additions of the initial matrix elements. In case of square matrices, the size of which is $n\times n$ , the number of the executed operations is the order $O(n^3)$.  We will assume further that all matrices are square and their sizes are $n\times n$.

\subsection{Sequential algorithm} % (fold)
\label{sub:sequential_algorithm}
The sequential matrix multiplication algorithm includes three nested loops:

\begin{lstlisting}
for (i=0; i<Size; i++){
  for (j=0; j<Size; j++){
    MatrixC[i][j] = 0;
    for (k=0; k<Size; k++){
      MatrixC[i][j] = MatrixC[i][j] + MatrixA[i][k]*MatrixB[k][j];
    }
  }
}
\end{lstlisting}

As each result matrix element is a scalar product of the initial matrix A row and the initial matrix B column, it
is necessary to carry out $n^2(2n-1)$ operations to compute all elements of the matrix $C$. As a result the time complexity of matrix multiplication is
\begin{equation}
  T_1 = n^2(2n-1)\tau_F
\end{equation}

% subsection sequential_algorithm (end)


% section matrix_matrix_multiplication (end)
